\section{Method}

\subsection{Momentum descent approach}
The momentum descent approach is a variant \texttt{an adjustment??} of the more known stochastic gradient descent. It and all its variant are probably the most used optimization algorithms for machine learning.\\
This algorithm is a gradient-based optimization. The main idea of this type of algorithm is to minimize a loss function (or cost/error function) following the advice\texttt{(direction??)} given by the gradient computed in the current point of the function.
\\
What can be a problem in the gradient descent is the large amount of example that at every iteration it must compute. Quite the opposite\texttt{Troppo informale?} is the stochastic gradient descent where there is an online training. Unfortunately\texttt{Troppo informale?}, the SGD is not as much stable as the classical GD, but requires only the evaluation of one example for \texttt{every??} iteration. 
Form now on, when we use talk about SDG, we refers to SDG with a subset of training dataset, called mini-batch or batch SDG.
\\
The convergence rate is of $O(\frac{1}{\sqrt{k}})$ (after k steps) when the algorithm is applied to a convex problem, if it is applied to a strongly convex problem the convergence rate is of $O(\frac{1}{k})$. (Mettere ref a libro di deeplearning)
\\
Momentum descent approach use the method of momentum to accelerate learning. It introduce a variable \textbf{v} in that we take account of the speed and direction at witch the parameters move through parameter space. The role of momentum is determined by the hyperparameter $\alpha\in[0,1)$.
The update rule is the following:
\begin{equation}
\label{classical_momentum}
\textbf{v}_k = \alpha\textbf{v}_{k-1} + \eta\nabla\textit{L}(\textbf{W}_k).
\end{equation}
\begin{equation}
\label{update_momentum}
\textbf{W}_k = \textbf{W}_{k-1}  + \textbf{v}_k.
\end{equation}
The Nesterov momentum is a variant of the classical Momentum, the main idea is to add a correction factor to the Nesterov momentum,  for this reason they differs in place where the gradient is evaluated. With Nesterov momentum the rate of convergence goes from $O(\frac{1}{k})$ to $O(\frac{1}{k^2})$.
The update rule for the velocity vector \textbf{v} is the following:
\begin{equation}
\label{nesterov_momentum}
\textbf{v}_k = \alpha\textbf{v}_{k-1} + \eta\nabla\textit{L}(\textbf{W}_k + \alpha\textbf{v}_{k-1}).
\end{equation}
\subsubsection{Algorithm}
\begin{algorithm}[H]
	\caption{Stochastic Gradient Descent Algorithm. The learning rate $\eta$, the $\alpha$ term and the maximum number of iterations are given.}
	\label{alg:sgd}
	\begin{algorithmic}[1]
		\Require{Learning rate $\eta$ and momentum parameter $\alpha$}
		\Require{Maximum number of iteration and error threshold}
		\Procedure{Stochastic Gradient Descent}{}
		\State Initialize \textbf{W} and \textbf{v}
		\State $k \gets 0$
		\While {$k < max\_iterations$ \&\& $error\_th<e$}
		\State Sample a mini-batch of \textit{m} training examples \{\textit{$(x_0,y_0),(x_1,y_1),...,(x_m,y_m)$}\}
		\If {Nesterov Momentum}
		\State $\tilde{\textbf{W}} \gets \textbf{W} + \alpha \textbf{v}$
		\EndIf
		\State Compute gradient estimate: $\textbf{g} \gets \frac {1}{m} \nabla \sum_i\textit{L}(\tilde{\textbf{W}})$
		\State Compute velocity update: $\textbf{v} \gets \alpha \textbf{v} - \eta \textbf{g}$
		\State Apply update: $\textbf{W} \gets \textbf{W} + \textbf{v}$
		\EndWhile
		\EndProcedure
	\end{algorithmic}
\end{algorithm}



 algorithm What is its complexity?

Is the algorithm(if it is iterative) guaranteed to converge? Is it going to be stable and return a good approximation of the solution(if it is direct)?  Are there any relevant convergence results?

Are the hypotheses of these convergence results (convexity, compactness, differentiability, etc.) satisfied by your problem? If not, what are the“closest” possible results you have available, and why exactly they are not applicable?  Do you expect this to be relevant in practice?

\subsection{Limited-memory quasi-Newton methods}
Limited-memory quasi-Newton methods are useful for solving large problems whose Hessian matrices cannot be computed at a reasonable cost or are not sparse.  
These methods maintain simple and compact approximations of Hessian matrices: instead of storing fully dense \textit{n} $\times$ \textit{n} approximations, they save only a few vectors of length \textit{n} that represent an approximation of it. 
 
The main idea of the L-BFGS algorithm is to use curvature information from only the most recent iterations to construct the Hessian approximation. Curvature information from earlier iterations, which is less likely to be relevant to the actual behavior of the Hessian at the current iteration is discarded in the interest of saving storage. 

\subsubsection{Limited memory BFGS}
L-BGFS algorithm derive from the BFGS method. So we begin our description of the L-BGFS method by recalling BFGS. 
Each step of the BGFS method has the form
\begin{equation}
\label{stepBFGS}
x_{k+1}=x_{k}-\alpha_{k}H_{k}\nabla f_{k} ,
\end{equation}
where $\alpha_{k}$ is the step length and $H_{k}$
is updated at every iteration by means of the formula

\begin{equation}
H_{k+1} = V^T_{k}H_{k}V_{k}+ \rho_{k}s_{k}s^T_{k}
\end{equation}
where
	 

 algorithm What is its complexity?

\subsubsection{Algorith Convergence}
\texttt{ Is the algorithm(if it is iterative) guaranteed to converge? Is it going to be stable and return a good approximation of the solution(if it is direct)?  Are there any relevant convergence results?}

Limited-memory variants  of  the  quasi-Newton  approach use  Hessian  approximations  that  can  be stored compactly by using just a few vectors of length \textit{n}. These methods are fairly robust, inexpensive, and easy to implement, but they do not converge rapidly.

 
  Are the hypotheses of theseconvergence results (convexity, compactness, differentiability, etc.) satisfied by your problem? If not, what are the“closest” possible results you have available, and why exactly they arenotapplicable?  Do you expect this to berelevant in practice?

\subsection{Bundle methods}


Next, we expect a brief recall of the algorithmic properties that you expect to see in the experiments. Is the algorithm(if it is iterative) guaranteed to converge? Is it going to be stable and return a good approximation of the solution(if it is direct)? What is its complexity? Are there any relevant convergence results? Are the hypotheses of theseconvergence results (convexity, compactness, differentiability, etc.) satisfied by your problem? If not, what are the“closest” possible results you have available, and why exactly they arenotapplicable?  Do you expect this to berelevant in practice?Again, you are advised to send us a version of this section by e-mail as soon as it is done. Again, we do not wantto see code at this point.