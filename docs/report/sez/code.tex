%\section{Code}

%Coding the algorithms is a major part of the project. Languages that are often convenient for numerical computationare (depending on the task) Matlab, Python, and C/C++, but we impose no strict requirements; you can use anyreasonable programming language.You are expected to implement the algorithm yourself; it shouldnotbe a single line of library call. However,you can use the numerical libraries of your language of choice for some of the individual steps: for instance, you canusenumpy.linalg.normto evaluate the error, or Matlab’sA \ bto solve a linear system that appears as a sub-step(unless, of course, writing a linear solver to compute that solution is a main task in your project).You can (and should) also use numerical libraries to compare their results to yours: for instance, checking if youralgorithm is faster or slower than Matlab’squadprog, if it produces (up to a tolerance) the same objective value, orhow the residual of your solution compares with that produced byA \ b.When in doubt if you should use a library, feel free to ask us.Your goal for this project is implementing and testing numerical algorithms: software engineering practices such asa full test suite, or pages of production-quality documentation, arenotrequired. That said, we appreciate well-writtenand well-documented code (who doesn’t?). You are free to use tools such asgitto ease your work, if you are familiarwith them (but giving us a pointer to thegitrepository is not the expected way to communicate with us).